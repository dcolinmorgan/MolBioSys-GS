\section{Discussion}
\label{discussion}
\gs is a package for GRN inference analysis, for method choice, evaluation and optimization.
It allows the user to control both network and data properties, and contains methods for exploring and analyzing these properties, as well as those of the inference method and its performance.
This opens up new possibilities to understand how these properties affect the performance of inference algorithms.

\gs can create networks with different topological properties that are typical for biological networks.
Given a network, \gs can then simulate expression data that is as informative as typical biological datasets.
However, its unique strength is varying these properties across a wide range of values to gain insights about the limitations and expected performance of a given inference method.
After all, biological data from different sources varies in terms of network and data properties, and the performance of inference methods depend on these properties.
Therefore, when a network inference method is applied to a biological dataset, it is important to make sure that the method has been benchmarked on simulated data with the same properties.
This is not standard practice today, even though it would provide information about the reliability of the inferred network.
%
\gs can also be used to investigate how network properties such as topology and IAA affect data properties.
The nature of the generated data also depends on the experimental design employed for perturbation.
Therefore, \gs supports a range of different experimental design schemes.
For instance, one can control the number of experiments and how many genes are perturbed in each experiment.
This can be of great value when designing wet lab experiments--if a particular design gives optimal results in a benchmark, then it should be the preferred choice when performing real experiments.

It has previously been observed that L$_1$ methods such as Glmnet perform considerably better at low than at high IAA   \citep{Tjarnberg2014}. Such a trend was not clearly observable in this benchmark, probably because the low IAA setting here was not as low as in the previous study.

The \gs package is implemented in \matlab and provides native storage functionality, as well as export options to formats more easily acceptable by other programming languages, including XML and JSON.
\gs fills a gap between theoretical analysis and experimental setup and could be incorporated in many current GRN inference and benchmarking pipelines.

\section{Acknowledgements}
\label{acknowledgements}
This work was partly supported by the Swedish strategic research program eSSENCE, a startup grant by the National Cheng Kung University, and a grant  from the Ministry of science and technology in Taiwan (105-2218-E-006-016-MY2).